\documentclass[../Cookbook.tex]{subfiles}


\begin{document}

\begin{recipe}{Spiced Pancakes}{Serves 2}{30 minutes prep + 15 minutes cooking}
	\Ingredient{1 cup (150g) flour (any desired mix of white and whole wheat)}
	\Ingredient{\fr32 tablespoon buttermilk powder\footnotemark}
	\Ingredient{1 teaspoon baking powder}
	\Ingredient{\fr14 teaspoon baking soda}
	\Ingredient{\fr14 teaspoon cloves}
	\Ingredient{\fr12 teaspoon cinnamon}
	\Ingredient{\fr18 teaspoon nutmeg}
	\Ingredient{\fr18 teaspoon ginger}
	Mix together all dry ingredients.
	You can mix several recipes of these together and keep for later in an airtight container.
	If so, one recipe is 180g
	%(about 19/3 ounces)
	of mix.

	\Ingredient{2 tablespoon butter}
	Melt butter in a separate mixing bowl. Start the griddle heating up.

	\Ingredient{\milk{1}}
	\Ingredient{1 egg}
	\Ingredient{\fr12 teaspoon vanilla}
	To the butter, add the milk, then the egg. Mix together.

	Combine wet and dry ingredients. Add more milk as needed to wet the whole mixture.

	\Ingredient{\fr12 can pumpkin pur\'ee, \fr12 cup mini chocolate chips, extra \fr14 teaspoon cloves, extra \milk{1/4}}
	\Ingredient{1 cup diced apple, heated}
	\Ingredient{1 cup blueberries, fresh or frozen then heated}
	\Ingredient{1 mashed banana}
	Add whatever extras you want to the batter. Some possibilities are shown here.
	Blueberries and apple dices should be heated, otherwise the water in them will absorb too much heat and will keep the batter around them from cooking properly.

	If the consistency is too thick, add more milk. The batter should be very liquid, and if you scoop some up and let it drip with the rest, it should swiftly spread out and become level again.

	\newstep
	Cook on the stovetop on a non-stick surface at a little less than Medium.
	Flip when edges are no longer liquid.
	The second side takes about half as long as the first side to finish.
\end{recipe}
\footnotetext{Can be substituted with \fr12 teaspoon cream of tartar}
\end{document}
