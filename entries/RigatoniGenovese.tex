\documentclass[../Cookbook.tex]{subfiles}

\begin{document}

\begin{recipe}{Rigatoni Alla Genovese}{3 servings}{1.5 hours}
% https://www.reddit.com/r/food/comments/4br9gt/like_onions_dont_have_plans_to_kiss_anyone_in_the/d1bqibk/

\Ingredient{Olive oil}
\Ingredient{Salt}
\Ingredient{1 kilogram yellow onions, finely chopped}
Liberally coat the bottom of a heavy bottom sauce pot with olive oil. Add all the onions, coat with more olive oil and a liberal amount of salt. Stir to combine. Cover and heat on medium for 10 minutes, stirring infrequently.

\Ingredient{300--400 grams ground beef or pork}
Once the onions are completely translucent and a good amount of liquid has appeared, stir in the meat and garlic. Stir long and hard to completely break apart the meat and combine it with the onions. Cover and simmer on very low for 1 hour, stirring every 10 minutes or so.

\Ingredient{125 milliliters dry white wine}
Deglaze the pot with the wine, stir to combine and turn the heat to medium-low to cook off the wine while the pasta cooks.

\Ingredient{225 grams rigatoni}
\Ingredient{60 milliliters table cream}
Bring a pot of salted water to boil, cook the rigatoni til el dente. As soon as you've added the rigatoni to the water reduce sauce heat back to very low and stir the cream into the sauce and adjust seasoning with salt and pepper.

\newstep
Preheat your broiler after you add the rigatoni to the water. Drain the pasta and add it to the sauce, stirring to combine.

\Ingredient{60 milliliters shaved parmesan}
Spoon the pasta onto plates and cover each with parmesan.
Place each plate under the broiler for 30--45 seconds, remove and garnish with parsley. Serve with extra parmesan on the side.


\end{recipe}

\end{document}
