\documentclass[../Cookbook.tex]{subfiles}

\begin{document}

\begin{recipe}[PieCrust]{Pie Dough}{2 crusts}{25 minutes prep, 30+ minutes chilling}

\Ingredient{\flour{all-purpose}{5}{2}; if you have very fine \ww, you can substitute 1 cup (150 grams)}
\Ingredient{1 teaspoon white sugar or 1 tablespoon powdered sugar}
\Ingredient{1 teaspoon salt; \fr12 teaspoon if using salted butter}
Mix together dry ingredients in a large bowl.

\Ingredient{1 cup cold shortening or butter}
If using a food processor, the fat should be freezer temperature. If cutting it in by hand, it should be refrigerator temperature.
Add the fat to the mixture.\\
Cut it in, either with two crossed knives or a pastry blender. Periodically mix and scrape your implements.
Stop when most of the fat is the consistency of coarse crumbs, but there remain some pea-sized pieces.
The mixture should seem dry and powdery, not greasy at all.

\Ingredient{\fr13 cup plus 1 tablespoon ice water}
Drizzle the water over the flour and fat mixture in parts. Mix in with a rubber spatula until the mixture looks evenly moistened and begins to form small balls.
If the balls stick together, you have added enough water; else, add a little more ice water. In dry climates like Colorado, you might need up to 3 tablespoons extra.\\
Continue mixing with the spatula, then clump the dough together with your hands.\\
Divide the dough into halves, then wrap them in plastic and refrigerate for at least 30 minutes.

\newstep
When ready, evenly roll out the dough to the desired thinness.

\end{recipe}

\end{document}
